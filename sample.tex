\documentclass[a4paper,draft]{book}
\usepackage{epsf,nano2cmr}
\begin{document}

\pnum{}
\ttitle{Exchange interaction of carriers with magnetic ions in novel
ZnMnSe/BeMnTe heterostructures with a type-II band alignment}

\tauthor{{\em D.~R.~Yakovlev},
A.~V.~Platonov, C.~Dem, W.~Ossau, L.~Hansen, A.~Waag, G.~Landwehr and  L.~W.~Molenkamp}

\ptitle{Exchange interaction of carriers with magnetic ions\\ in novel
ZnMnSe/BeMnTe heterostructures\\ with a type-II band alignment}

\pauthor{{\em D.~R.~Yakovlev}$^{1,2}$,
A.~V.~Platonov$^{2}$, C.~Dem$^{4}$, W.~Ossau$^{3}$, L.~Hansen$^{3}$,
A.~Waag$^{3,5}$, G.~Landwehr$^{3}$\\ and  L.~W.~Molenkamp$^{3}$}


\affil{$^{1}$~Fachbereich Physik, Universit\"{a}t Dortmund, 44221 Dortmund, Germany\\
$^{2}$~\FTI\\
$^{3}$~Physikalisches Institut, Universit\"{a}t W\"{u}rzburg, 97074 W\"{u}rzburg, Germany\\
$^{4}$~Institut f\"{u}r Physikalische Chemie, Universit\"{a}t W\"{u}rzburg, 97074 W\"{u}rzburg, Germany\\
$^{5}$~Abteilung Halbleiterphysik, Universit\"{a}t Ulm, 89081 Ulm, Germany}

\begin{abstract}
{(Zn,Mn)Se/(Be,Mn)Te diluted-magnetic-semiconductor
heterostructures with a type-II band alignment have been fabricated by
molecular-beam epitaxy.
Giant Zeeman splitting of the band states, caused by their interaction with the localized magnetic
moments of Mn ions, has been observed for the spatially direct- and indirect optical transitions.
The value of the $p{-}d$ exchange interaction constant of holes with magnetic Mn-ions
in (Be,Mn)Te $N_{0}\beta=-0.40\pm 0.05$~eV has been measured.}
\end{abstract}

\begindc

\index{Yakovlev D. R.}
\index{Platonov A. V.}
\index{Dem C.}
\index{Ossau W.}
\index{Hansen L.}
\index{Waag A.}
\index{Landwehr G.}
\index{Molenkamp L. W.}


\section*{Introduction}

(Be,Mn)Te is a new material in the family of II--VI diluted
magnetic semiconductors (DMS).
Published information on magnetic,
optical and electronic properties of (Be,Mn)Te is very limited.
Magneto-luminescence experiments confirm the giant Zeeman
splitting for holes located in (Be,Mn)Te layers of ZnSe/(Be,Mn)Te
hetero\-structures~[1,~2].
The rough evaluation of the $p{-}d$ exchange
interaction constant $N_{0}\beta = -0.4$~eV  has been
done on the basis of these experiments.

A special interest in (Be,Mn)Te material exists due to its very high p-type
dopability.
In BeTe free hole concentrations of more than $1\times 10^{20}$~cm$^{-3}$
has been reported~[3], which in turn could lead to
ferromagnetic alignment of the Mn spins mediated by free holes, similar to
(Ga,Mn)As.
Such ferromagnetic ordering is an important prerequisite for the
manipulation of spin-oriented carriers without permanent external magnetic
fields.
Indeed, ferromagnetic behavior of heavily p-type doped (Be,Mn)Te
could recently be demonstrated by transport measurements~[4].

ZnSe/BeTe is a heterosystem with a type-II band\linebreak alignment.
Its main specific feature is a large conduction (about 2~eV) and valence (about 1~eV) band
offset, which results in a very small penetration of the carrier wave
functions into the neighboring layers.
The type-II nature of the interface allows to separately access the Zeeman splitting of valence-band states in
(Be,Mn)Te by investigation of (Zn,Mn)Se/(Be,Mn)Te heterostructures~[1].
In the present work we exploit this advantage of (Zn,Mn)Se/(Be,Mn)Te structures
to measure the constant of $p{-}d$ exchange coupling of free holes in the valence
band of (Be,Mn)Te with the spins of Mn ions.

\section {Experimental}

Results for three Zn$_{1-x}$Mn$_{x}$Se/Be$_{1-y}$Mn$_{y}$Te multiple\linebreak
quantum well structures (MQW) grown by MBE on (100) GaAs substrates are
reported here.
These structures have the same design and consist of ten
pairs of 200~{\AA}/100~{\AA} Zn$_{1-x}$Mn$_{x}$Se/Be$_{1-y}$Mn$_{y}$Te.
They differ by the Mn content (see data in Table).
A Mn effusion cell in the MBE chamber was open during the growth of all layers, Mn content in
(Be,Mn)Te layers was 6.8 times higher that in (Zn,Mn)Se layers due to the
difference in their growth rates (see details in Ref.~[2]).
Structure parameters are summarized in Table.

Photoluminescence (PL) was excited by UV lines of a {\it cw} Ar-ion laser.
Polarized PL spectra were recorded at a temperature of 1.6~K with a
charge-coupled-device (CCD).
Magnetic fields up to 7.5~T were applied in the
Faraday geometry parallel to the structure growth axis.
As a result two emission lines appear in PL spectra of these structures with a type-II band
alignment:
(i)~The spatially direct emission (D) at about 2.8~eV, which
involves electrons and holes recombining in (Zn,Mn)Se layers, and
(ii)~spatially indirect emission (ID) at about 1.8~eV caused by electrons and
holes from (Zn,Mn)Se- and (Be,Mn)Te layers, respectively.

\section {p{-}d exchange constant in BeMnTe}
\begin{figure}[b]
\leavevmode \centering{\epsfbox{sample01.eps}}
\caption[]{Photoluminescence spectra of a
Zn$_{0.99}$Mn$_{0.01}$Se/ Be$_{0.932}$Mn$_{0.068}$Te MQW recorded
with and without magnetic field applied in the Faraday geometry.
Pronounced giant Zeeman shifts are seen both for direct and
indirect emission lines.}
\end{figure}

One can see from the scheme in Fig.~1 that in (Zn,Mn)Se/ (Be,Mn)Te
structures the giant Zeeman splitting of the indirect optical
transition (ID) is the sum of the splittings in the conduction
band of (Zn,Mn)Se and in the valence band of (Be,Mn)Te.
By measuring the giant Zeeman splitting of the direct optical
transition (D) in (Zn,Mn)Se layers, one can get the value of the
conduction-band splitting in these layers, and therefore get the
pure Zeeman splitting of the (Be,Mn)Te valence band.
The later we need to determine the $p{-}d$ exchange constant.

In Fig.~1 PL spectra of the direct and indirect emission of sample {\#}2 are
shown for magnetic fields of 0~T and 7.5~T.
The Zeeman shift is very pronounced both for direct and indirect optical transitions.
The behavior of the direct emission is identical to that of MBE-grown (Zn,Mn)Se epilayers
studied in great detail in~[5].
We use it to determine exact Mn content
in the grown samples on the base of common phenomenological approach
suggested by Gaj \etal~[6].
In this approach the energies of the conduction
band ($E_{\rm C}$) and the valence band ($E_{\rm V}$) states, interacting via strong
exchange coupling with the localized spins of Mn-ions oriented by an
external magnetic field $B$, can be described by the following equations:
\begin{eqnarray}
E_{\rm C}   &=&
xN_0\alpha s_e \langle {S_z } \rangle,
\quad s_e = \pm 1/2
\\
E_{\rm V}   &=&
\frac{1}{3}xN_0 \beta J \langle {S_z } \rangle,
\quad J=\pm 3/2; \pm 1/2
\end{eqnarray}
where
$$
\langle {S_z } \rangle = S_{\rm eff}
B_{5/2}\left( {\frac{5\mu_{\rm B} g_{\rm Mn} B}{2k_{\rm B} \left(T + T_0 \right)}} \right)
$$
is described by the modified Brillouin function, which
accounts phenomenologically for the Mn-Mn $d{-}d$ interaction by
introducing two parameters: the effective Mn spin $S_{\rm eff} $ and
the effective temperature $T_0 $.
The giant Zeeman shift of the heavy-hole states in Be$_{1-y}$Mn$_{y}$Te layers
in the studied heterostructures is plotted in Fig.~2.
It was obtained from the Zeeman shift of indirect emission line.
From this shift the contribution of electrons in (Zn,Mn)Se
was substracted.
\begin{figure}[b]
\leavevmode \centering{\epsfbox{sample02.eps}}
\caption{Giant Zeeman splitting of hole states in Be$_{1 -y}$Mn$_{y}$Te
determined from magnetic-field shift of the indirect
emission line for (Zn,Mn)Se/(Be,Mn)Te MQWs. 50{\%} of the
splitting is plotted. Experimental data are shown by symbols.
Lines are the fitting curves to evaluate the exchange constant
$N_0 \beta _2 $ for Be$_{1 - y}$Mn$_{y}$Te (see also Table).
$T=1.6$~K.}
\end{figure}
A linewidth of the indirect emission band is
considerably larger than that of the direct one.
That causes relatively large error bar for data points, especially in the
sample {\#}1 with the smallest shift.
Fitting of the experimental dependencies from Fig.~2 allows us to evaluate the exchange
constant $N_0 \beta _2 $ for Be$_{1-y}$Mn$_{y}$Te.
For this fit done with Eq.~(2) and shown by solid lines the exchange constant
and the temperature were taken as a free parameter.
The Mn-ion concentration $y$ in the Be$_{1-y}$Mn$_{y}$Te layers has been
taken from the Mn content of (Zn,Mn)Se layers corrected by a
factor 6.8 arising from the difference in the growth rates.
For $T_0 $ and $S_{\rm eff} $ parameters calibrated dependencies of
(Zn,Mn)Se have been used.
$N_0 \beta _2 $ and $T$ parameters providing the best fit are given in the Table.
The values of $N_0 \beta_2 $ vary from $-0.30$ to $-0.40$~eV, measured with the accuracy
of about $\pm 0.05$~eV.
One can see that for samples {\#}1 and {\#}3 the Mn temperature deviates from the lattice temperature,
which evidences the weak heating of the Mn ion system.
For the sample {\#}2 the temperature is very close to the lattice
temperature and therefore we consider these data as the most
reliable for the evaluation of the $p{-}d$ exchange constant.
The value $N_0 \beta_2 = -0.40\pm 0.05$~eV for (Be,Mn)Te is about two
times smaller than $-0.88$~eV in (Cd,Mn)Te and $-1.05$~eV in (Zn,Mn)Te.
Theoretical approach developed by Larson \etal{}
[7] predicts the value of --1.05~eV for the (X,Mn)Te-containing
alloys, independent of the cation X character.
A considerable deviation of the experimental value in (Be,Mn)Te from the
theoretical expectations is stimulating for revising and/or
refining this model.

\begin{table}[h]
\vspace{-12pt}
\caption{}
\begin{center}
\tabcolsep4pt
\begin{tabular}
{lcccc}
\hline
                &$x$,       &$y$,       &$T$ (K)    &$N_0 \beta _2$  \\
Sample          &ZnMnSe     &BeMnTe     &BeMnTe     &BeMnTe\\
\hline
\vspace{4pt}
{\#}1           & 0.005     & 0.034     & 4         &$-0.30$ $\pm $ 0.07 \\
%\hline
{\#}2           & 0.010     & 0.068     & 2         &$-0.40$ $\pm $ 0.05 \\
%\hline
{\#}3           & 0.014     & 0.095     & 3.3       &$-0.35$ $\pm $ 0.05 \\
\hline
\end{tabular}
\end{center}
\vspace{-12pt}
\end{table}

Interesting to note here that due to the type-II band alignment of
the studied structures the transition matrix element for the
optical transition, which is indirect in real space and involves
conduction-band electrons from ZnSe layers and va\-len\-ce-band holes
from BeTe layers, extremely ``interface sensitive''. This causes
very strong optical anisotropy to be decoded by polarized optical
spectroscopy~[8].

\ack
This work has been supported in part by the Deutsche
For\-schungs\-ge\-mein\-schaft through SFB 410.

\begin{thebibliography}{8}
\itemsep-2pt

\bibitem{yakdr1}
D.~R.~Yakovlev \etal,
{\em Appl. Phys. Lett.} {\bf 78}, 1870 (2001).

\bibitem{yakdr2}
D.~R.~Yakovlev \etal,
{\em Proc. 26th ICPS}, Edinburgh, UK 2002, publication on CD.

\bibitem{yakdr3}
H.~J.~Lugauer \etal,
{\em J. Crystal Growth} {\bf 175/176}, 619 (1997).

\bibitem{yakdr4}
L.~Hansen \etal,
{\em Appl. Phys. Lett.} {\bf 79}, 3125 (2001).

\bibitem{yakdr5}
W.~Y.~Yu \etal,
{\em Phys. Rev. B} {\bf 51}, 9722 (1995).

\bibitem{yakdr6}
J.~A.~Gaj \etal,
{\em Solid State Commun.} {\bf 29}, 435 (1979).

\bibitem{yakdr7}
B.~E.~Larson \etal,
{\em Phys. Rev. B} {\bf 37}, 4137 (1988).

\bibitem{yakdr8}
D.~R.~Yakovlev \etal,
{\em Phys. Rev. Lett.} {\bf 88}, 257401 (2002).

\end{thebibliography}
\end{document}
